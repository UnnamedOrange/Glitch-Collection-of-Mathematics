% Licensed under the Creative Commons Attribution Share Alike 4.0 International.
% See the LICENCE file in the repository root for full licence text.

\documentclass[UTF8]{book}

\title{数学罔论}
\author{UnnamedOrange}

\usepackage{ctex}
\usepackage{hyperref}
\hypersetup
{
	pdfauthor = {UnnamedOrange},
	pdftitle = {数学罔论}
}

\usepackage{amsmath}
\usepackage{amsthm}
\usepackage{amsfonts}
\usepackage{amssymb}
\usepackage{graphicx}
\usepackage{xcolor}
\usepackage{xifthen}
\usepackage[normalem]{ulem}
\usepackage[framemethod=TikZ]{mdframed}
\mdfsetup{hidealllines=true,%
	skipabove=\topskip,%
	innertopmargin=\topskip, innerbottommargin=\topskip,%
	everyline=true, splittopskip=1.75\topskip}

\usepackage{makeidx}
\makeindex

% 特殊符号
\newcommand \comma {,}

% 数集
\newcommand \N {\mathbb N}
\newcommand \Q {\mathbb Q}
\newcommand \R {\mathbb R}
\newcommand \Z {\mathbb Z}
\newcommand \C {\mathbb C}

% index
\newcommand \idx[1] {#1\index{#1}}

% def
\makeatletter
\newcommand \indexer[1]
{%
	\@for\@myvar:={#1}\do{\index{\@myvar}}%
}
\makeatother
\newenvironment{definition}[1]
{
	\definecolor{shadecolor}{RGB}{233, 233, 243}
	\begin{mdframed}[backgroundcolor=shadecolor]\relax
	\indexer{#1}
	\noindent
	\textbf{定义} \emph{#1}:%
}
{%
	\end{mdframed}
}

% theorem
\newcounter{cTheorem}
\newenvironment{theorem}[1][]
{
	\definecolor{shadecolor}{RGB}{243, 233, 233}
	\begin{mdframed}[backgroundcolor=shadecolor]\relax
	\noindent
	\textbf{定理}
	\ifthenelse{\isempty{#1}}{\stepcounter{cTheorem}\arabic{cTheorem}}{\emph{#1}\index{#1}}:%
}
{%
	\end{mdframed}
}

% proof
\renewenvironment{proof}[1][]
{
	\definecolor{shadecolor}{RGB}{237, 237, 237}
	\begin{mdframed}[backgroundcolor=shadecolor]\relax
	{\noindent \textbf{证明}~{#1}}

}
{%
	\end{mdframed}
}

% solve
\newenvironment{solve}[1][]
{
	\definecolor{shadecolor}{RGB}{237, 237, 237}
	\begin{mdframed}[backgroundcolor=shadecolor]\relax
	{\noindent \textbf{解}~{#1}}

}
{%
	\end{mdframed}
}