% Licensed under the Creative Commons Attribution Share Alike 4.0 International.
% See the LICENCE file in the repository root for full licence text.

\section{复数域}

\subsection{从复数集到复数域}

\begin{definition}{复数,复数集,虚数单位}
	称集合 $\{x + y \mathrm i: x, y \in \R\}$ 为\emph{复数集},记为 $\C$。其中 $\mathrm i$ 是一个符号,称为\emph{虚数单位}。复数集中的任一元素称为一个\emph{复数}。
\end{definition}

显然,$\C$ 中的任一元素 $x + y \mathrm i$ 与 $\R \times \R$ 中的元素 $(x, y)$ 一一对应。

规定当 $y = 0$ 时,复数 $x + y \mathrm i$ 就是实数集中的元素 $x$,即:
$$
(x + 0 \mathrm i) = x \in \R
$$
所以实数集是复数集的真子集。

当 $x = 0$ 且 $y \ne 0$ 时,复数 $x + y \mathrm i$ 可以简记为 $y \mathrm i$。特别地,当 $x = 0, y = 1$ 时,复数可以只记为 $\mathrm i$。

\begin{definition}{实部,虚部}
	设复数 $z = x + y \mathrm i$,则称 $x$ 为 $z$ 的\emph{实部},记为 $\operatorname{Re} z$;称 $y$ 为 $z$ 的\emph{虚部},记为 $\operatorname{Im} z$。
\end{definition}

定义了实部和虚部后,我们可以得到以下显然的定理。

\begin{theorem}[复数相等]
	复数 $z_1, z_2$ 相等的充分必要条件是它们的实部和虚部分别相等。
\end{theorem}

最后,让我们考察复数的运算。

\begin{definition}{复数域}
	定义加法:
	$$
	\begin{aligned}
		{+} \colon & \C \times \C \to \C
		\\&
		(x_1 + y_1 \mathrm i, x_2 + y_2 \mathrm i) \mapsto (x_1 + x_2) + (y_1 + y_2) \mathrm i
	\end{aligned}
	$$

	定义乘法:
	$$
	\begin{aligned}
		{\times} \colon & \C \times \C \to \C
		\\&
		(x_1 + y_1 \mathrm i, x_2 + y_2 \mathrm i) \mapsto (x_1 x_2 - y_1 y_2) + (x_1 y_2 + x_2 y_1) \mathrm i
	\end{aligned}
	$$

	则 $(\C, {+}, {\times})$ 是一个域,称之为\emph{复数域}。
\end{definition}

复数域的存在\footnote{有关域的概念超出了本书的讨论范围,这里我们直接给出了 $(\C, {+}, {\times})$ 是一个域的结论。},为复数集注入了无限的活力。借此,我们可以立即给出以下定义:
\begin{enumerate}
	\item \emph{相反数}。设复数 $z = x + y \mathrm i$,则它存在唯一的相反数 $-x - y \mathrm i$,记为 $-z$。
	\item \emph{减法}。
	$$
	\begin{aligned}
		{-} \colon & \C \times \C \to \C
		\\&
		(z_1, z_2) \mapsto z_1 + (-z_2)
	\end{aligned}
	$$
	\item \emph{倒数}。设复数 $z = x + y \mathrm i \pod{z \ne 0}$,则存在唯一复数 $z_1$ 满足 $z z_1 = 1$,称 $z_1$ 为 $z$ 的倒数。

	$z$ 的倒数记为:
	$$
	\dfrac{1}{z} = \dfrac{1}{x + y \mathrm i}
	$$

	该记法与实数运算中所见的分式相同。因为 $z z_1 = 1$,所以该记法\textbf{满足约分通分法则}。
	$$
	\dfrac{1}{x + y \mathrm i} = \dfrac{x - y \mathrm i}{(x + y \mathrm i)(x - y \mathrm i)} = \dfrac{x - y \mathrm i}{x^2 + y^2}
	$$
	\item \emph{除法}。
	$$
	\begin{aligned}
		{\div} \colon & \C \times (\C \backslash \{0\}) \to \C
		\\&
		(z_1, z_2) \mapsto z_1 \times \dfrac{1}{z_2}
	\end{aligned}
	$$
\end{enumerate}

事实上,复数减法和加法互为逆运算,复数除法和乘法互为逆运算。

\subsection{数域}

数域是域的一个极端特例。\textbf{数域中元素的运算法则与我们熟知的一致},因此我们只研究数域中元素的集合。

\begin{definition}{数域}
	复数集的一个子集 $\mathbb K$ 如果满足:
	\begin{enumerate}
		\item $0, 1 \in \mathbb K$;
		\item $a, b \in \mathbb K \Longrightarrow a \pm b, ab \in \mathbb K$;
		\item $a, b \in \mathbb K \pod{b \ne 0} \Longrightarrow \dfrac{a}{b} \in \mathbb K$。
	\end{enumerate}

	则称 $(\mathbb K, {+}, {\times})$ 是一个\emph{数域},简写为 $\mathbb K$ 是一个数域。
\end{definition}

根据定义,$\C$ 是最大的数域,即所有数域均为 $\C$ 的子集。我们熟知的 $\R, \Q$ 均为数域,但 $\Z$ 不是数域,因为它对除法运算不封闭。

我们不加证明地给出下面的定理。

\begin{theorem}
	$\Q$ 是最小的数域,即 $\Q$ 是所有数域的子集。
\end{theorem}

\begin{theorem}
	有无穷多个数域 $\mathbb F$ 满足 $\Q \subset \mathbb F \subset \R$。
\end{theorem}

\begin{theorem}
	不存在数域 $\mathbb F$ 满足 $\R \subset \mathbb F \subset \C$。
\end{theorem}

\section{复平面}

前文提到,$\C$ 中的任一元素 $x + y \mathrm i$ 与 $\R \times \R$ 中的元素 $(x, y)$ 一一对应,于是复数 $x + y \mathrm i$ 可以与平面直角坐标系中的点 $(x, y)$ 一一对应。

\begin{definition}{复平面,实轴,虚轴}
	若平面直角坐标系中的各点 $(x, y)$ 与复数 $x + y \mathrm i$ 一一对应,则称该平面为\emph{复平面}。复平面的 $x$ 轴称为\emph{实轴},$y$ 轴称为\emph{虚轴}。
\end{definition}