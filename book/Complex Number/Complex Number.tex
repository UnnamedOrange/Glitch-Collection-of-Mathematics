% Licensed under the Creative Commons Attribution Share Alike 4.0 International.
% See the LICENCE file in the repository root for full licence text.

\section{复数域}

\begin{definition}{复数,复数集,虚数单位}
	称集合 $\{x + y \mathrm i: x, y \in \R\}$ 为\emph{复数集},记为 $\C$。其中 $\mathrm i$ 是一个符号,称为\emph{虚数单位}。复数集中的任一元素称为一个\emph{复数}。
\end{definition}

显然,$\C$ 中的任一元素 $x + y \mathrm i$ 与 $\R \times \R$ 中的元素 $(x, y)$ 一一对应。

规定当 $y = 0$ 时,复数 $x + y \mathrm i$ 就是实数集中的元素 $x$,即:
$$
(x + 0 \mathrm i) = x \in \R
$$
所以实数集是复数集的真子集。

当 $x = 0$ 且 $y \ne 0$ 时,复数 $x + y \mathrm i$ 可以简记为 $y \mathrm i$。特别地,当 $x = 0, y = 1$ 时,复数可以只记为 $\mathrm i$。

\begin{definition}{实部,虚部}
	设复数 $z = x + y \mathrm i$,则称 $x$ 为 $z$ 的\emph{实部},记为 $\operatorname{Re} z$;称 $y$ 为 $z$ 的\emph{虚部},记为 $\operatorname{Im} z$。
\end{definition}

定义了实部和虚部后,我们可以得到以下显然的定理。

\begin{theorem}[复数相等]
	复数 $z_1, z_2$ 相等的充分必要条件是它们的实部和虚部分别相等。
\end{theorem}

最后,让我们考察复数的运算。

\begin{definition}{复数域}
	定义加法:
	$$
	\begin{aligned}
		{+} \colon & \C \times \C \to \C
		\\&
		(a_1 + b_1 \mathrm i, a_2 + b_2 \mathrm i) \mapsto (a_1 + a_2) + (b_1 + b_2) \mathrm i
	\end{aligned}
	$$

	定义乘法:
	$$
	\begin{aligned}
		{\times} \colon & \C \times \C \to \C
		\\&
		(a_1 + b_1 \mathrm i, a_2 + b_2 \mathrm i) \mapsto (a_1 a_2 - b_1 b_2) + (a_1 b_2 + a_2 b_1) \mathrm i
	\end{aligned}
	$$

	则 $(\C, {+}, {\times})$ 是一个域,称之为\emph{复数域}。
\end{definition}

复数域的存在\footnote{有关域的概念超出了本书的讨论范围,这里我们直接给出了 $(\C, {+}, {\times})$ 是一个域的结论。},为复数集注入了无限的活力。